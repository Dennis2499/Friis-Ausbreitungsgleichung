% !TeX root = ./Protokoll.tex
\usepackage[ngerman]{babel}
\usepackage[acronym, shortcuts=ac, toc]{glossaries-extra}
\usepackage{graphicx}
\usepackage{hyperref}
\usepackage{float}
\usepackage{microtype}
\usepackage[utf8]{inputenc}
\usepackage{csquotes}
\usepackage{etoolbox}
\usepackage[default]{sourcesanspro}
\usepackage[T1]{fontenc}

\usepackage[
  backend=biber,
  style=numeric-comp,
  sorting=none,
  backref,
  maxbibnames=5,
  toc=bib,
  giveninits=true,
  defernumbers=true,
  urldate=long
]{biblatex}

\makeatletter
\hypersetup
{
  pdftitle = {\@title},
  pdfsubject = {Laborversuch},
  pdfauthor = {\@author},
  unicode = {true},
  %  bookmarks = {true},
  bookmarksopen = {true},
  hidelinks
}
\makeatother

\makeatletter
\AtBeginDocument{
  \vspace*{-1.5cm}
  \includegraphics[width=.3\linewidth]{logos/ditlogo-schwarz-de.pdf}\hfill\includegraphics[width=.4\linewidth]{logos/HTWK-H-DE-black.pdf}\\\\

  {\sffamily
  \begin{center}
		\Large \textbf{\@title}
  \end{center}

  \begin{tabbing}
    
    Gruppenmitglieder::: \= blablabla blablblabla \kill \\
    Erarbeitet von: \> \@author{}\\
    2. Gruppenmitglied: \> \grouppartner{}\\
	Gruppennummer: \> \groupnumber{}\\
    Versuchsdatum: \> \labdate \\    
    \end{tabbing}
	
\vspace{-1,5em}
	\hrule
}
}
\makeatother
